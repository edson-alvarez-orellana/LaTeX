\documentclass[menu.tex]{subfiles}
\graphicspath{ {images/} }
\begin{document}    
\begin{tabular} {p{3cm} p{4.5cm} p{9cm}}
\multicolumn{3}{c}{\begin{LARGE}Menú Semanal 5\end{LARGE}}\\
\hline

%---LUNES---%
\pbox{20cm}
{
    \rule{0pt}{3ex}\begin{large}\textbf{Lunes}\end{large}\\ 
    \rule{0pt}{2ex}Panqueque light
}& 
\vspace{-0.6cm}
\begin{compactitem} 
    \begin{scriptsize}
        \item Pechuga de pollo troceada
        \item 1 palta maduro pelado
        \item 1 manzana pelada
        \item \nicefrac{1}{4} taza de apio
        \item \nicefrac{1}{2} taza de cebolla
        \item Perejil o cilantro
        \item 2 cucharillas de limón
        \item Sal
        \item Pimienta negra molida
        \item Aceite de oliva
    \end{scriptsize}
\end{compactitem}&
\vspace{-0.6cm} 
Mezclar 2 cucharadas de salvado de avena, 1 chorro de leche, 1 cuchara de avena normal, \% clara de huevo, una pizca de Stevia o sal y cocinar lentamente en teflón.\\
\hline

%---MARTES---%
\pbox{20cm}
{
    \rule{0pt}{3ex}\begin{large}\textbf{Martes}\end{large}\\ 
    \rule{0pt}{2ex}Pastel de jamon
}& 
\vspace{-0.5cm}            
\begin{compactitem} 
    \begin{scriptsize}
        \item 1 taza de fresas troceadas
        \item 1 taza de manzana troceada
        \item 1 taza de uvas en mitades
        \item 1 taza de naranja troceada
        \item 2 aguacates troceados
        \item 1 taza de nueces
        \item 1 taza de pechuga de pollo
        \item 1 taza de tocino troceado
        \item 1 taza de cebolla troceada
        \item Queso rallado
        \item Aceite de oliva               
    \end{scriptsize}
\end{compactitem}&
\vspace{-0.5cm}
Coloque en un plato una tira de jamón de pollo o pavo. Cubra con una lonja de queso. Sobre el queso disperse champiñones y aceitunas (eliminando las semillas). Coloque nuevamente una tira de jamón y otras de queso, cubriendo esta última con rodajas de tomate. Nuevamente jamón y queso, cubriendo con carne molida cocida o pollo desmenuzado, otra tira de jamón, cubra con una delgada capa de salsa de tomate y orégano. Hornee en el microondas, puede agregar yogurt natural o salsa césar light entre capas a fin de conseguir un pastel más cremoso.\\
\hline

%---MIERCOLES---%
\pbox{20cm}
{
    \rule{0pt}{3ex}\begin{large}\textbf{Miércoles}\end{large}\\
    \rule{0pt}{2ex}Atun tropical
}&
\vspace{-0.8cm}
\begin{compactitem} 
    \begin{footnotesize}
        \item Tortillas integrales o Lechuga
        \item Carne molida
        \item Tomate
        \item Cebolla
        \item Palta
    \end{footnotesize}
\end{compactitem}&
\vspace{-0.8cm}
Mezclar en un envase 1 lata de atún al agua, 2 chorros de mostaza, 2 cucharadas de yogurt natural, apio y perejil picados. Servir con hojas de lechuga, tomate y cebolla. Si tiene habilitada la piña, agregue trocitos pequeños en cantidad moderada.\\
\hline

%---JUEVES---%
\pbox{20cm}
{
    \rule{0pt}{3ex}\begin{large}\textbf{Jueves}\end{large}\\
    \rule{0pt}{2ex}Preparaciones de pollo al yogurt, a la Coca-Cola y a la naranja
} & 
\vspace{-0.8cm}
\begin{compactitem} 
    \begin{footnotesize}
        \item Carote berenjenas o brócoli
        \item Huevo
        \item Jamón
        \item Queso
    \end{footnotesize}
\end{compactitem}&
\vspace{-0.8cm}
Prepare y sazone el pollo permitidos para su persona (ver sector sazonadores del formulario, no use otros). Bañe con yogurt natural, coca cola Zero o jugo de naranja según lo desee, hornee y a media cocción vuelque las presas y báñelas nuevamente con el ingrediente elegido (yogurt natural, pollo a la coca cola o jugo de naranja) termine de hornear.\\
\hline

%---VIERNES---%
\pbox{20cm}
{
    \rule{0pt}{3ex}\begin{large}\textbf{Viernes}\end{large}\\ 
    \rule{0pt}{2ex}Pimenton relleno
} & 
\vspace{-0.3cm}
\begin{compactitem} 
    \begin{footnotesize}
        \item \nicefrac{1}{2} kg de trigo pelado
        \item \nicefrac{1}{2} kg de papas
        \item \nicefrac{1}{4} kg carne de res
        \item Cuerito de cerdo un pedazo
        \item 2 cucharas de colorante rojo
        \item \nicefrac{1}{2} taza de habas
        \item \nicefrac{1}{4} taza de arvejas
        \item 2 cebollas medianas
        \item 1 zanahoria rallada
        \item 2 ramitas de perejil picado
        \item 2 dientes de ajo molidos
        \item \nicefrac{1}{4} cucharilla de comino
        \item 4 cucharas de aceite
        \item 6 litros de agua
        \item Sal a gusto
        \item 1 cuchara de perejil picado
    \end{footnotesize}
\end{compactitem}&
\vspace{-0.3cm}
Corte la parte superior del pimentón y retire el interior, rellene con carne molida cocinada previamente con queso, cebolla, tomate y champiñones, hornee.
\\
\hline

%---SABADO---%
\pbox{20cm}
{
    \rule{0pt}{3ex}\begin{large}\textbf{Sábado}\end{large}\\
    \rule{0pt}{2ex}Crema de verduras con champiñones y queso
}& 
\vspace{-0.3cm}
\begin{compactitem} 
    \begin{footnotesize}
        \item Verduras
        \item Huevo
        \item Queso
        \item Jamón o enrollado de pollo
        \item Aceitunas
        \item Carne en trocitos de pollo
        \item Salsa cesar light
    \end{footnotesize}
\end{compactitem}&
\vspace{-0.3cm}
prepare una sopa con las verduras permitidas y carne, licue las verduras con el caldo, A la crema resultante agréguele champiñones (pueden ser los de lata), queso rallado y la carne en trozos.\\ \hline
\newpage
\end{tabular}
\end{document}
