\documentclass[menu.tex]{subfiles}
\graphicspath{ {images/} }
\begin{document}
\begin{tabular} {p{3cm} p{4.5cm} p{9cm}}
\multicolumn{3}{c}{\begin{LARGE}Menú Semanal 3\end{LARGE}}\\
\hline

 %---LUNES---%
\pbox{20cm}
{
    \rule{0pt}{3ex}\begin{large}\textbf{Lunes}\end{large}\\ 
    \rule{0pt}{2ex}Atún con \\cebollines
}&
\vspace{-0.5cm}
\begin{compactitem} 
    \begin{footnotesize}
        \item \nicefrac{1}{2} lata de atún
        \item 1 manojo de cebollin picado
        \item 1 manojo de perejil picado
        \item Mayonesa light
        \item Lechuga
    \end{footnotesize}
\end{compactitem}&
\vspace{-0.5cm}
Mezcla media lata de atún al agua con los cebollines picados a gusto (recomendable una cantidad generosa). Agregar perejil picado a gusto (para un sabor potente se recomienda una buena cantidad) y mayonesa light. Mezclar todos los ingredientes, servir sobre una hoja de lechuga y acompañar con otros vegetales.\\
\hline

%---MARTES---%
\pbox{20cm}
{
    \rule{0pt}{3ex}\begin{large}\textbf{Martes}\end{large}\\ 
    \rule{0pt}{2ex}Sopa de maní
} & 
\vspace{-0.4cm}
\begin{compactitem} 
    \begin{footnotesize}
        \item 6 nudos de carne de vaca
        \item 1 cuchara de sal molida
        \item 6 papas
        \item \nicefrac{1}{2} taza de arvejas
        \item \nicefrac{1}{2} taza de habas verdes
        \item 1 taza de maní molido
        \item 3 litros de agua para cocer la carne
        \item 1 taza de cebolla
        \item \nicefrac{1}{2} taza de tomate
        \item 1 cucharilla de comino
        \item 2 dientes de ajo
        \item 1 cuchara de perejil
        \item 1 \nicefrac{1}{2} cuchara de ají amarillo molido
    \end{footnotesize}
\end{compactitem}&
\vspace{-0.4cm}
Parta las papas peladas en cuatro, pele las arvejas, pique la cebolla menuda y el tomate pelado.
Fría la cebolla en el aceite. En una olla ponga al fuego los 3 litros de agua.
Antes que empiece a hervir, agréguele la carne. 
Deje dar un hervor y ponga la sal, el tomate, la cebolla, el comino, el orégano, 
el ají y el ajo retostados en aceite. Luego agregue el maní molido. 
Deje cocer hasta que la carne quede blanda y cocida. Agregue las habas, 
las arvejas y las papas. Cocine hasta que estén suaves.
Sirva en plato hondo, con un pedazo de carne y adorne con el perejil.\\
\hline

%---MIERCOLES---%
\pbox{20cm}
{
    \rule{0pt}{3ex}\begin{large}\textbf{Miércoles}\end{large}\\ 
    \rule{0pt}{2ex}Ensalada de \\coliflor
} & 
\vspace{-0.6cm}
\begin{compactitem} 
    \begin{scriptsize}
        \item \nicefrac{1}{2} cabeza de coliflor
        \item \nicefrac{1}{4} cebolla
        \item 2 tallos de apio
        \item 2 huevos cocidos
        \item Pepinillos
        \item Mayonesa (mejor casera)
        \item 1 diente de ajo machacado
        \item 1 cucharilla de mostaza
        \item Pimienta negra
    \end{scriptsize}
\end{compactitem}&
\vspace{-0.6cm}
Corta la coliflor en cogollos y cuécela en una olla a presión. Ten cuidado de no cocerla demasiado para que no quede como una pasta.
Una vez cocida, escúrrela y enfríala bajo el grifo. Sécala bien y ponla sobre papel de cocina absorbente para secarla aún más. Reserva.
Trocea la cebolla, el apio, los huevos y los pepinillos y reserva.
Desmenuza la coliflor en un bol y añade el resto de ingredientes, incluidos la mayonesa, el ajo, la mostaza y la pimienta.
Mezcla todo bien y sirve.\\
\hline

%---JUEVES---%
\pbox{20cm}
{
    \rule{0pt}{3ex}\begin{large}\textbf{Jueves}\end{large}\\ 
    \rule{0pt}{2ex}Jak’a lawa
} & 
\vspace{-0.4cm}
\begin{compactitem} 
    \begin{footnotesize}
        \item \nicefrac{1}{2} Kg carne de res
        \item \nicefrac{1}{2} Kg de papas
        \item 5 litros de caldo
        \item 4 choclos molidos
        \item 2 cebollas picadas
        \item 5 dientes de ajo molidos
        \item 1 tomate picado
        \item 1 zanahoria rallada
        \item 1 nabo pequeño rallado
        \item 2 ramitas de perejil
        \item 2 ajíes rojo molidos
        \item \nicefrac{1}{4} cucharilla de comino
        \item \nicefrac{1}{2} taza de habas
        \item \nicefrac{1}{4} taza de arvejas
        \item 1 cuchara de perejil
        \item Queso raspado
\end{footnotesize}
\end{compactitem}&
\vspace{-0.4cm}
1. Poner en una olla el agua al fuego, cuando empiece a hervir añadir la carne.
2. Aparte, calentar el aceite en la sartén, sofreír la cebolla con el ajo hasta que doren, 
incorporar, el comino, el tomate, la zanahoria, el perejil y el ají, mezclar y seguir cocinando.
3. Vaciar en la olla donde se están cocinado las carnes el ahogado junto con las arvejas, 
dejar cocer 30 minutos.
4. Añadir las papas, las habas, el choclo molido, el orégano y la sal, sin dejar de mover 
hasta que vuelva a hervir y estén cocidas las papas.
5. Lo típico es servir en platos de barro (cerámica) con cucharas de palo (madera) echando 
encima perejil picado y queso rallado. \\
\hline

%---VIERNES---%
\pbox{20cm}
{
    \rule{0pt}{3ex}\begin{large}\textbf{Viernes}\end{large}\\
    \rule{0pt}{2ex}Berenjenas \\rebozadas
} & 
\vspace{-0.6cm}
\begin{compactitem} 
    \begin{footnotesize}
        \item Berenjenas
        \item Huevo
        \item Sal
        \item Pimienta
        \item Aceite de oliva
    \end{footnotesize}
\end{compactitem}&
\vspace{-0.6cm}
Corte las berenjenas en lonjas y úntelas en una preparación que incluya huevo, sal y una pizca de pimienta, cocínelas en un poco de aceite de oliva o en aerosol.\\
\hline

%---SABADO---%
\pbox{20cm}
{
    \rule{0pt}{3ex}\begin{large}\textbf{Sábado}\end{large}\\ 
    \rule{0pt}{2ex}Sopa de \\papaliza
} & 
\vspace{-0.5cm}
\begin{compactitem} 
    \begin{footnotesize}
        \item \nicefrac{1}{2} Kilo de costilla de res
        \item 1 Pedazo de charque
        \item 3 Tazas de papalisa cocida y machucada
        \item 3 Cucharadas de aceite
        \item \nicefrac{1}{2} Taza de habas
        \item \nicefrac{1}{2} Taza de arvejas
        \item 1 Cebolla
        \item 2 Ramas de hierba buena
        \item 3 Papas
        \item Sal a gusto
        \item 2 Litros de agua
    \end{footnotesize}
\end{compactitem}&
\vspace{-0.5cm}
Pon los dos litros de agua en una olla y hazla hervir.
Cuando este en plena ebullición agrégale la carne de res y la charque. Déjalas cocer a fuego moderado.
En una sartén calienta un poco el aceite y refrié la papalisa, agrega las arvejas, las habas y la cebolla picada a la pluma.
Pon la papalisa en el caldo donde se esta cociendo la carne y déjala cocer muy bien. Desde que la papalisa este cocida agrega las papas peladas y el resto en el sartén.
Espera a que todo haya cocido muy bien para sazonar con sal y servir con hierba buena picada fina encima para adornar. \\
\hline

\newpage
\end{tabular}
\end{document}
