\documentclass[menu.tex]{subfiles}
\graphicspath{ {images/} }
\begin{document}    
\begin{tabular} {p{3cm} p{4.5cm} p{9cm}}
\multicolumn{3}{c}{\begin{LARGE}Acompañamientos para el té\end{LARGE}}\\
\hline

\pbox{20cm}
{
    \rule{0pt}{2ex}Cupcakes \\dietéticos
}& 
\vspace{0.5px}
\begin{compactitem} 
    \begin{scriptsize}
        \item Cda. polvo de hornear
        \item Cda. canela en pollo
        \item 1 taza de linaza molida
        \item Stevia
        \item Huevo
    \end{scriptsize}
\end{compactitem}&
\vspace{0.5px}
Mezclar en una taza de porcelana \nicefrac{1}{2} cucharilla de polvo de hornear, \nicefrac{1}{2} cucharada de canela en polvo, \nicefrac{1}{4} taza de linaza molida y Stevia a gusto. Agregar 1 huevo entero hasta lograr una masa semisólida y colocar al microondas de 2 a 3 mín. El tiempo necesario para su cocción dependerá de cada microondas).\\
\hline

\pbox{20cm}
{
    \rule{0pt}{2ex}Pan dietético
}& 
\vspace{0.5px}        
\begin{compactitem} 
    \begin{scriptsize}
        \item Cda. de polvo de hornear
        \item Cda. mantequilla
        \item Sal
        \item Huevo
    \end{scriptsize}
\end{compactitem}&
\vspace{0.5px}
Mezclar en un Tupper plástico 1 cucharilla de polvo para hornear, \nicefrac{1}{2} taza molida, 1 cucharilla de
mantequilla y sal salud al 50\% a gusto. Agregar 1 huevo entero hasta lograr una masa semisólida y colocar al microondas de 2 a 3 mín. (el tiempo necesario para su cocción dependerá de cada microondas).
\\
\hline

\pbox{20cm}
{
    \rule{0pt}{2ex}Humintas
}&
\vspace{0.5px}
\begin{compactitem} 
    \begin{footnotesize}
        \item 12 unidades de choclo
        \item Albahaca seca
        \item Leche deslactosada light
        \item 5 Cds. aceite de oliva        
        \item Canela molida
        \item Anis
        \item Sal
        \item Stevia
    \end{footnotesize}
\end{compactitem}&
\vspace{0.5px}
Licuar los granos de choclo (12 unidades si son pequeños o 6 si son grandes) junto con la albahaca seca. Si los choclos son muy secos o maduros agregar un poco de leche deslactosada light. Agregar a la preparación 5 cucharas de aceite de oliva o light más canela molida, anís, Stevia y/o sal. Batir y mezclar la preparación hasta que quede homogénea. Poner en la sartén previamente cubierta con una capa de aceite anti adherente de canola o con aceite de oliva. La preparación debe tener la altura de un dedo.
Cocinar primero con fuego fuerte, luego a fuego lento. Cuando desprende se requiere dar la vuelta viendo que esté dorado. Terminar de cocer la preparación. Al retirarla es recomendable colocar la huminta en papel absorbente.\\
\hline

\pbox{20cm}
{
    \rule{0pt}{2ex}Panqueque \\de canela
} & 
\vspace{0.5px}
\begin{compactitem} 
    \begin{footnotesize}
        \item 6 claras de huevo
        \item 1 taza de avena
        \item Canela molida
        \item Stevia
        \item Aceite de oliva
    \end{footnotesize}
\end{compactitem}&
\vspace{0.5px}
Licuar 6 claras de huevo con 1 taza de avena, agregar una pizca de canela molida y Stevia. Cocinar el panqueque en la sartén precalentada con aceite anti adherente o un mínimo de aceite de oliva, al retirar de la sartén colocar en papel absorbente.\\
\hline

\newpage
\end{tabular}
\end{document}
