\documentclass[menu.tex]{subfiles}
\graphicspath{ {images/} }
\begin{document}    
\begin{tabular} {p{3.5cm} p{4cm} p{9cm}}
\multicolumn{3}{c}{\begin{LARGE}Menú Semanal 1\end{LARGE}}\\
\hline

%---LUNES---%
    \pbox{20cm}
    {
        \rule{0pt}{3ex}\begin{large}\textbf{Lunes}\end{large}\\ 
        \rule{0pt}{2ex}Ensalada de pollo con \\ palta\\
        \includegraphics[scale=0.35]{pollo-con-palta} 
    } & 
    \vspace{-2cm}            
    \begin{compactitem} 
        \begin{scriptsize}
            \item Pechuga de pollo troceada
            \item 1 palta maduro pelado
            \item 1 manzana pelada
            \item \nicefrac{1}{4} taza de apio
            \item \nicefrac{1}{2} taza de cebolla
            \item Perejil o cilantro
            \item 2 cucharillas de limón
            \item Sal
            \item Pimienta negra molida
            \item Aceite de oliva
        \end{scriptsize}
    \end{compactitem}&
    \vspace{-2cm}
    Corta la pechuga de pollo en trozos pequeños y sofríe en una sartén con un poco de aceite de oliva. Reserva. Trocea el aguacate, la manzana, el apio y la cebolla. Reserva. Coloca en un bol los trocitos de pollo, aguacate, manzana, apio y cebolla. Con la ayuda de un tenedor machaca el aguacate y mézclalo con el resto de ingredientes hasta formar una especie de pasta. Añade el perejil (o cilantro), el zumo de limón (o lima), la sal y la pimienta y mézclalo bien. Puedes añadir más zumo o incluso aceite de oliva si ves que está demasiado seco. Sirve acompañado de pan, pan tostado o algún tipo de hoja verde para ensalada, como lechuga, berros, verdolaga. Nota: puedes conservarlo en el frigorífico tapado con film transparente.\\
    \hline

%---MARTES---%
    \pbox{20cm}
    {
        \rule{0pt}{3ex}\begin{large}\textbf{Martes}\end{large}\\ 
        \rule{0pt}{2ex}Ensalada vegetal con carne\\
        \includegraphics[scale=0.33]{ensalada-vegetal-con-carne} 
    } & 
    \vspace{-1.75cm}            
    \hspace{0.5cm}\begin{footnotesize}Ingredientes (ensalada)\end{footnotesize}
    \begin{compactitem} 
        \begin{scriptsize}
            \item 1 taza de fresas troceadas
            \item 1 taza de manzana troceada
            \item 1 taza de uvas en mitades
            \item 1 taza de naranja troceada
            \item 2 aguacates troceados
            \item 1 taza de nueces
            \item 1 taza de pechuga de pollo
            \item 1 taza de bacón troceado
            \item 1 taza de cebolla troceada
            \item Queso gorgonzola rallado
            \item Aceite de oliva               
        \end{scriptsize}
    \end{compactitem}
    \hspace{0.5cm}\begin{footnotesize}Ingredientes (aderezo)\end{footnotesize}
    \begin{compactitem} 
        \begin{scriptsize}
            \item 2 limones maduros
            \item 4 dientes de ajo machacados
            \item \nicefrac{1}{2} taza de aceite de oliva
            \item \nicefrac{1}{4} taza de miel
            \item 1 cucharilla de mostaza
            \item Vino tinto
            \item Sal
        \end{scriptsize}
    \end{compactitem} &
    \vspace{-1.7cm}
    Elaboración (ensalada)
    Fríe en una sartén con aceite de oliva la pechuga de pollo y el bacón troceados.
    En un bol grande, mezcla el pollo y el bacón con el resto de ingredientes. Reserva.
    Elaboración (aderezo)
    Exprime los limones en un bol aparte y añade el ajo, el aceite de oliva, la miel, la mostaza, el vinagre y la sal. Si el gusto no te convence, añade más sal o miel para adaptarlo a tu paladar.    Sirve la ensalada en platos individuales y deja que cada comensal se aliñe su ensalada con la cantidad de aderezo que desee.\\
    \hline

%---MIERCOLES---%
\pbox{20cm}
{
\rule{0pt}{3ex}\begin{large}\textbf{Miércoles}\end{large}\\
\rule{0pt}{2ex}Tacos en base de \\ tortillas y en base \\ vegetal \\ \includegraphics[scale=0.17]{tacos-integrales}
}&
\vspace{-2cm}
\begin{compactitem} 
\begin{footnotesize}
\item Tortillas integrales o Lechuga
\item Carne molida
\item Tomate
\item Cebolla
\item Palta
\end{footnotesize}
\end{compactitem}&
\vspace{-2cm}
Sobre una tortilla integral coloque carne molida cocinada, vegetales, tomate, cebolla y un poco de palta. Puede preparar los tacos sobre 2 hojas grandes de lechuga, de este modo se reduce significativamente la cantidad de calorías.\\
\hline

%---JUEVES---%
\pbox{20cm}
{
\rule{0pt}{3ex}\begin{large}\textbf{Jueves}\end{large}\\
\rule{0pt}{2ex}Pastel de carote,\\ brócoli o berenjenas\\
\includegraphics[scale=0.42]{pastel-brocoli}
} & 
\vspace{-1.6cm}
\begin{compactitem} 
\begin{footnotesize}
\item Carotes o berenjenas o brócoli
\item Huevo
\item Jamón
\item Queso
\end{footnotesize}
\end{compactitem}&
\vspace{-1.6cm}
Coloque en una bandeja una fila de carotes o berenjenas cortados en rodajas o brócoli cocido picado. Cubra esta capa con huevo, coloque jamón, vegetales y queso rallado. Cubra nuevamente con carotes, berenjenas o brócoli, De este modo coloque capas hasta formar un pastel del tamaño que le agrade y hornee.\\
\hline

%---VIERNES---%
\pbox{20cm}
{
\rule{0pt}{3ex}\begin{large}\textbf{Viernes}\end{large}\\
\rule{0pt}{2ex}Berenjenas rebozadas\\
\includegraphics[scale=0.13]{berenjenas_rebozadas}
} & 
\vspace{-1.8cm}
\begin{compactitem} 
\begin{footnotesize}
\item Berenjenas
\item Huevo
\item Sal
\item Pimienta
\item Aceite de oliva
\end{footnotesize}
\end{compactitem}&
\vspace{-1.8cm}
Corte las berenjenas en lonjas y úntelas en una preparación que incluya huevo, sal y una pizca de pimienta, cocínelas en un poco de aceite de oliva o en aerosol.\\
\hline

%---SABADO---%
\pbox{20cm}
{
\rule{0pt}{3ex}\begin{large}\textbf{Sábado}\end{large}\\
\rule{0pt}{2ex}Ensalada Cesar\\
\includegraphics[scale=0.40]{ensalada-cesar}
}& 
\vspace{-1.6cm}
\begin{compactitem} 
\begin{footnotesize}
\item Verduras
\item Huevo
\item Queso
\item Jamon o enrollado de pollo
\item Aceitunas
\item Carne en trocitos de pollo
\item Salsa cesar light
\end{footnotesize}
\end{compactitem}&
\vspace{-1.6cm}
Mezcle las verduras de su preferencia con huevos, queso en cuadrados, jamón o enrollado de pollo, aceitunas y carne en trocitos o pollo desmenuzado, coloque salsa cesar light (de cualquier marca) para decorar y sazonar. \\ \hline
\newpage
\end{tabular}
\end{document}
