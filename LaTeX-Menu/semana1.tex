\documentclass[menu.tex]{subfiles}
\graphicspath{ {images/} }
\begin{document}    
\begin{tabular} {p{3cm} p{4.5cm} p{9cm}}
\multicolumn{3}{c}{\begin{LARGE}Menú Semanal 1\end{LARGE}}\\
\hline

%---LUNES---%
\pbox{20cm}
{
    \rule{0pt}{3ex}\begin{large}\textbf{Lunes}\end{large}\\ 
    \rule{0pt}{2ex}Ensalada de pollo\\ con palta
}& 
\vspace{-0.6cm}
\begin{compactitem} 
    \begin{scriptsize}
        \item Pechuga de pollo troceada
        \item 1 palta maduro pelado
        \item 1 manzana pelada
        \item \nicefrac{1}{4} taza de apio
        \item \nicefrac{1}{2} taza de cebolla
        \item Perejil o cilantro
        \item 2 cucharillas de limón
        \item Sal
        \item Pimienta negra molida
        \item Aceite de oliva
    \end{scriptsize}
\end{compactitem}&
\vspace{-0.6cm} 
Corta la pechuga de pollo en trozos pequeños y sofríe en una sartén con un poco de aceite de oliva. Reserva. Trocea el aguacate, la manzana, el apio y la cebolla. Reserva. Coloca en un bol los trocitos de pollo, aguacate, manzana, apio y cebolla. Con la ayuda de un tenedor machaca el aguacate y mézclalo con el resto de ingredientes hasta formar una especie de pasta. Añade el perejil (o cilantro), el zumo de limón (o lima), la sal y la pimienta y mézclalo bien. Puedes añadir más zumo o incluso aceite de oliva si ves que está demasiado seco. Sirve acompañado de pan, pan tostado o algún tipo de hoja verde para ensalada, como lechuga, berros, verdolaga. Nota: puedes conservarlo en el frigorífico tapado con film transparente.\\
\hline

%---MARTES---%
\pbox{20cm}
{
    \rule{0pt}{3ex}\begin{large}\textbf{Martes}\end{large}\\ 
    \rule{0pt}{2ex}Ensalada vegetal\\ con carne
}& 
\vspace{-0.5cm}            
\hspace{0.5cm}\begin{footnotesize}Ingredientes (ensalada)\end{footnotesize}
\begin{compactitem} 
    \begin{scriptsize}
        \item 1 taza de fresas troceadas
        \item 1 taza de manzana troceada
        \item 1 taza de uvas en mitades
        \item 1 taza de naranja troceada
        \item 2 aguacates troceados
        \item 1 taza de nueces
        \item 1 taza de pechuga de pollo
        \item 1 taza de tocino troceado
        \item 1 taza de cebolla troceada
        \item Queso rallado
        \item Aceite de oliva               
    \end{scriptsize}
\end{compactitem}
\hspace{0.5cm}\begin{footnotesize}Ingredientes (aderezo)\end{footnotesize}
\begin{compactitem} 
    \begin{scriptsize}
        \item 2 limones maduros
        \item 4 dientes de ajo machacados
        \item \nicefrac{1}{2} taza de aceite de oliva
        \item \nicefrac{1}{4} taza de miel
        \item 1 cucharilla de mostaza
        \item Vino tinto
        \item Sal
    \end{scriptsize}
\end{compactitem}&
\vspace{-0.5cm}
Elaboración (ensalada)
Fríe en una sartén con aceite de oliva la pechuga de pollo y el tocino troceados.
En un bol grande, mezcla el pollo y el tocino con el resto de ingredientes. Reserva.

Elaboración (aderezo)
Exprime los limones en un bol aparte y añade el ajo, el aceite de oliva, la miel, la mostaza, el vinagre y la sal. Si el gusto no te convence, añade más sal o miel para adaptarlo a tu paladar. Sirve la ensalada en platos individuales y deja que cada comensal se aliñe su ensalada con la cantidad de aderezo que desee.\\
\hline

%---MIERCOLES---%
\pbox{20cm}
{
    \rule{0pt}{3ex}\begin{large}\textbf{Miércoles}\end{large}\\
    \rule{0pt}{2ex}Tacos en base de \\ tortillas y en base \\ vegetal
}&
\vspace{-0.8cm}
\begin{compactitem} 
    \begin{footnotesize}
        \item Tortillas integrales o Lechuga
        \item Carne molida
        \item Tomate
        \item Cebolla
        \item Palta
    \end{footnotesize}
\end{compactitem}&
\vspace{-0.8cm}
Sobre una tortilla integral coloque carne molida cocinada, vegetales, tomate, cebolla y un poco de palta. Puede preparar los tacos sobre 2 hojas grandes de lechuga, de este modo se reduce significativamente la cantidad de calorías.\\
\hline

%---JUEVES---%
\pbox{20cm}
{
    \rule{0pt}{3ex}\begin{large}\textbf{Jueves}\end{large}\\
    \rule{0pt}{2ex}Pastel de carote,\\ brócoli o \\berenjenas
} & 
\vspace{-0.8cm}
\begin{compactitem} 
    \begin{footnotesize}
        \item Carote berenjenas o brócoli
        \item Huevo
        \item Jamón
        \item Queso
    \end{footnotesize}
\end{compactitem}&
\vspace{-0.8cm}
Coloque en una bandeja una fila de carotes o berenjenas cortados en rodajas o brócoli cocido picado. Cubra esta capa con huevo, coloque jamón, vegetales y queso rallado. Cubra nuevamente con carotes, berenjenas o brócoli, De este modo coloque capas hasta formar un pastel del tamaño que le agrade y hornee.\\
\hline

%---VIERNES---%
\pbox{20cm}
{
    \rule{0pt}{3ex}\begin{large}\textbf{Viernes}\end{large}\\ 
    \rule{0pt}{2ex}Llusp’ichi
} & 
\vspace{-0.3cm}
\begin{compactitem} 
    \begin{footnotesize}
        \item \nicefrac{1}{2} kg de trigo pelado
        \item \nicefrac{1}{2} kg de papas
        \item \nicefrac{1}{4} kg carne de res
        \item Cuerito de cerdo un pedazo
        \item 2 cucharas de colorante rojo
        \item \nicefrac{1}{2} taza de habas
        \item \nicefrac{1}{4} taza de arvejas
        \item 2 cebollas medianas
        \item 1 zanahoria rallada
        \item 2 ramitas de perejil picado
        \item 2 dientes de ajo molidos
        \item \nicefrac{1}{4} cucharilla de comino
        \item 4 cucharas de aceite
        \item 6 litros de agua
        \item Sal a gusto
        \item 1 cuchara de perejil picado
    \end{footnotesize}
\end{compactitem}&
\vspace{-0.3cm}
Remojar el trigo pelado noche antes. Al día siguiente lavar en varias aguas, 
retirando la tierra y piedrecillas.
Poner al fuego una olla con el agua. Cuando empiece a hervir añadir las carne de res, 
el cuerillo de cerdo y el trigo. Aparte calentar el aceite en la sartén en el fuego y 
sofreír la cebolla con el ajo. Dejando dorar, incorporar el comino, el perejil, 
la zanahoria, el colorante y las arvejas. 
Añadir este ahogado a la olla y dejar cocer. Mover continuamente para que no se pegue 
al fondo de la olla, hasta que las carnes estén suaves y el trigo reviente, luego 
incorporar las papas, las habas. Sazonar con la sal y el orégano, al final si está muy 
líquido añadir un poco de harina diluida en agua sin grumos. Dejar cocer las papas y 
retirar del fuego. Servir caliente.\\
\hline

%---SABADO---%
\pbox{20cm}
{
    \rule{0pt}{3ex}\begin{large}\textbf{Sábado}\end{large}\\
    \rule{0pt}{2ex}Ensalada Cesar
}& 
\vspace{-0.3cm}
\begin{compactitem} 
    \begin{footnotesize}
        \item Verduras
        \item Huevo
        \item Queso
        \item Jamón o enrollado de pollo
        \item Aceitunas
        \item Carne en trocitos de pollo
        \item Salsa cesar light
    \end{footnotesize}
\end{compactitem}&
\vspace{-0.3cm}
Mezcle las verduras de su preferencia con huevos, queso en cuadrados, jamón o enrollado de pollo, aceitunas y carne en trocitos o pollo desmenuzado, coloque salsa cesar light (de cualquier marca) para decorar y sazonar.\\ \hline
\newpage
\end{tabular}
\end{document}
