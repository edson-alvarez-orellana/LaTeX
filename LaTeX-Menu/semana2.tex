\documentclass[menu.tex]{subfiles}
\graphicspath{ {images/} }
\begin{document}            
\begin{tabular} {p{3cm} p{4.5cm} p{9cm}}      
\multicolumn{3} { c }{\begin{LARGE}Menú Semanal 2\end{LARGE}}\\
\hline
    
%---LUNES---%
\pbox{20cm}
{
    \rule{0pt}{3ex}\begin{large}\textbf{Lunes}\end{large}\\ 
    \rule{0pt}{2ex}Silpancho
} & 
\vspace{-0.4cm}
\begin{compactitem} 
    \begin{footnotesize}
        \item Carne en filetes
        \item 1 taza de arroz graneado
        \item 4 papas cocidas
        \item 4 huevos
        \item Tomate, Cebolla y Locoto
        \item Pan molido
        \item Vinagre, Aceite
        \item Sal y Pimienta 
    \end{footnotesize}
\end{compactitem}&
\vspace{-0.4cm}
Sazonar la carne con sal y pimienta y adelgazar con el pan molido hasta que quede como un asado grande y delgado, se debe agregar pan molido constantemente y suficiente para que la carne no se rompa con los golpes.
Freír las papas previamente cocidas y cortadas en rodajas, los huevos y la carne apanada.
Picar el tomate, la cebolla y el locoto en cuadritos chicos; mezclar y sazonar con sal, pimienta y vinagre.
Servir en un plato poniendo una cama de arroz, distribuir unas cuantas papas, encima poner la carne, el huevo frito y decorar con la salsa cruda.\\
\hline 

%---MARTES---%
\pbox{20cm}
{
    \rule{0pt}{3ex}\begin{large}\textbf{Martes}\end{large}\\
    \rule{0pt}{2ex}Albóndigas \\rellenas
}&
\vspace{-0.6cm}
\begin{compactitem} 
    \begin{footnotesize}
        \item Albóndigas de res o atún
        \item Salvado de trigo
        \item Jamón
        \item Queso
        \item Huevo
    \end{footnotesize}
\end{compactitem}&
\vspace{-0.6cm}
Prepare las albóndigas tal como sabe hacerlas, solo que, en vez de usar pan molido, utilice salvado de trigo o de avena muy bien molidos (si la marca que compro no viene bien molida, licue el salvado en seco hasta que se parezca a la harina). Haga un hueco en la albóndiga y rellene con queso y jamón picados en forma fina, cocinarlos en aceite. Aplane la preparación sin rellenar y tendrás unas agradables hamburguesas dietéticas. Use poco huevo para no requerir mucho salvado ya que este al ser cocidos tiende a endurecerse. Puede probar la misma preparación en base a atún al agua.\\
\hline
    
%---MIERCOLES---%
\pbox{20cm}
{
    \rule{0pt}{3ex}\begin{large}\textbf{Miércoles}\end{large}\\ 
    \rule{0pt}{2ex}Pique Macho 
} & 
\vspace{-0.4cm}
\begin{compactitem} 
    \begin{footnotesize}
        \item 1 Kg de carne de res
        \item \nicefrac{1}{2} kilo de salchichas
        \item 8 papas medianas
        \item 4 cebollas picadas tipo pluma
        \item 2 tomates medianos
        \item 2 locotos picados
        \item 1 cda. de pimienta molida
        \item 1 pisca de comino molido
        \item 2 cucharaditas de sal
        \item Ajo molido (opcional)
        \item 3 huevos cocidos
        \item 1\nicefrac{1}{2} taza de aceite para freír
    \end{footnotesize}
\end{compactitem}&
\vspace{-0.4cm}
Cortar la carne en dados pequeños y condimentarla con sal, ajo, pimienta y comino a gusto.
Freír la carne tapando la cacerola para obtener una carne jugosa.
Aparte, freír las salchichas cortadas en rodajas.
Pelar y cortar las papas en bastones y freírlas en aceite caliente.
Mezclar la carne, las papas fritas y las salchichas.
Servir adornando con el locoto crudo picado en tiras largas, los huevos duros, la cebolla y el tomate. \\
\hline

%---JUEVES---%
\pbox{20cm}
{
    \rule{0pt}{3ex}\begin{large}\textbf{Jueves}\end{large}\\ 
    \rule{0pt}{2ex}Ensalada caprese\\ con pollo
} & 
\vspace{-0.6cm}
            
\begin{compactitem} 
    \begin{scriptsize}
        \item Bolitas de mozzarella marinadas
        \item Pechuga de pollo
        \item Tomates cherry
        \item Pimienta molida
        \item Sal
        \item Aceite de oliva
    \end{scriptsize}
\end{compactitem} &
\vspace{-0.6cm}     
Saca las bolitas de mozzarella de su marinada y vierte la mitad de la marinada en un bol.
Coloca el pollo en el bol con marinada, sazona con sal y pimienta y déjalo reposar durante 1 hora a temperatura ambiente.
Corta las bolitas de mozzarella, ponlas en un bol aparte y añade la otra mitad de marinada. Reserva.
Añade los tomates cherry al bol de la mozzarella y sazona con sal y pimienta. Reserva.
Cocina la pechuga de pollo a la plancha dejando que se tueste un poco y córtala en pequeños cubos.
Añade los trocitos de pollo al bol con la mozzarella y los tomates, mezcla y sirve.\\
\hline

%---VIERNES---%
\pbox{20cm}
{
    \rule{0pt}{3ex}\begin{large}\textbf{Viernes}\end{large}\\ 
    \rule{0pt}{2ex}Ensalada de \\brócoli y manzana
} & 
\vspace{-0.6cm}
\begin{compactitem} 
    \begin{footnotesize}
        \item 2 tazas de de brócoli
        \item 1 manzana pelada en cubitos
        \item Nueces tostadas y troceadas
        \item Fruta deshidratada
        \item 3 lonchas de tocino
        \item 1 cucharilla de miel
        \item Vinagre
        \item \nicefrac{1}{2} taza de mayonesa
        \item Sal
        \item Pimienta
        \item Aceite de oliva
    \end{footnotesize}
\end{compactitem}&
\vspace{-0.6cm}
Trocea el tocino y cocínelo a la plancha con unas gotas de aceite de oliva. Reserva.
Introduce el brócoli en agua hirviendo durante un minuto y enfríalo bajo el grifo. Escurre y reserva.
Coloca en un bol el brócoli, la manzana, las nueces, la fruta deshidratada y el tocino. Reserva.
En otro bol, mezcla la miel, el vinagre y la mayonesa hasta conseguir un aderezo de consistencia suave.
Sazona el aderezo con sal y pimienta.
Añade el aderezo al bol con el brócoli y el resto de ingredientes y mezcla bien.

Nota: puedes ajustar las cantidades de ingredientes usados para el aderezo según tus gustos.\\
\hline

%---SABADO---%
\pbox{20cm}
{
    \rule{0pt}{3ex}\begin{large}\textbf{Sábado}\end{large}\\ 
    \rule{0pt}{2ex}Ensalada de \\tomate, tocino \\y aguacate
} & 
\vspace{-0.6cm}
\begin{compactitem} 
    \begin{footnotesize}
        \item 1 aguacate maduro
        \item 1 tomate
        \item 2 huevos cocidos
        \item 3 lonchas de tocino
        \item Sal
        \item Pimienta
        \item Zumo de limón
        \item Aceite de oliva
    \end{footnotesize}
\end{compactitem}&
\vspace{-0.6cm}
Cocina el tocino a la plancha con unas gotas de aceite de oliva y córtalo en trozos muy pequeños. Reserva.
Trocea el aguacate, los huevos y el tomate. Reserva.
Exprime uno o dos gajos de limón para obtener el zumo.
Coloca el aguacate, los huevos, el tomate y el tocino en un bol.
Sazona con sal y pimienta y añade el zumo de limón.
Mézclalo todo bien y sirve.\\
\hline

\newpage        
\end{tabular}
\end{document}
