\documentclass[menu.tex]{subfiles}
\graphicspath{ {images/} }
\begin{document}            
\begin{tabular} {p{3.5cm} p{4cm} p{9cm}}        
\multicolumn{3} { c }{\begin{LARGE}Menú Semanal 2\end{LARGE}}\\
\hline
    
 %---LUNES---%
\pbox{20cm}
{
    \rule{0pt}{3ex}\begin{large}\textbf{Lunes}\end{large}\\ 
    \rule{0pt}{2ex}Atún con \\cebollines
}&
\vspace{-0.5cm}
\begin{compactitem} 
    \begin{footnotesize}
        \item \nicefrac{1}{2} lata de atún
        \item 1 manojo de cebollin picado
        \item 1 manojo de perejil picado
        \item Mayonesa light
        \item Lechuga
    \end{footnotesize}
\end{compactitem}&
\vspace{-0.5cm}
Mezcla media lata de atún al agua con los cebollines picados a gusto (recomendable una cantidad generosa). Agregar perejil picado a gusto (para un sabor potente se recomienda una buena cantidad) y mayonesa light. Mezclar todos los ingredientes, servir sobre una hoja de lechuga y acompañar con otros vegetales.\\
\hline
    
%---MARTES---%
\pbox{20cm}
{
    \rule{0pt}{3ex}\begin{large}\textbf{Martes}\end{large}\\
    \rule{0pt}{2ex}Albóndigas rellenas
}&
\vspace{-0.4cm}
\begin{compactitem} 
    \begin{footnotesize}
        \item Albóndigas de res o atún
        \item Salvado de trigo
        \item Jamón
        \item Queso
        \item Huevo
    \end{footnotesize}
\end{compactitem}&
\vspace{-0.4cm}
Prepare las albóndigas tal como sabe hacerlas, solo que, en vez de usar pan molido, utilice salvado de trigo o de avena muy bien molidos (si la marca que compro no viene bien molida, licue el salvado en seco hasta que se parezca a la harina). Haga un hueco en la albóndiga y rellene con queso y jamón picados en forma fina, cocinarlos en aceite. Aplane la preparación sin rellenar y tendrás unas agradables hamburguesas dietéticas. Use poco huevo para no requerir mucho salvado ya que este al ser cocidos tiende a endurecerse. Puede probar la misma preparación en base a atún al agua.\\
\hline
    
%---MIERCOLES---%
\pbox{20cm}
{
    \rule{0pt}{3ex}\begin{large}\textbf{Miércoles}\end{large}\\ 
    \rule{0pt}{2ex}Ensalada de pollo
} & 
\vspace{-0.4cm}            
\begin{compactitem} 
    \begin{scriptsize}
        \item 2 tazas de pollo troceado
        \item 1 taza de uvas rojas (u otro tipo) cortadas a la mitad
        \item 2 huevos cocidos y troceados
        \item Mayonesa (mejor casera)
        \item Un poquito de eneldo fresco
        \item 1 diente de ajo picado
        \item Sal
        \item Pimienta
    \end{scriptsize}
\end{compactitem}&
\vspace{-0.4cm}        
Fríe el pollo en una sartén con aceite de oliva. Reserva.
En un bol, coloca el pollo, añade el resto de ingredientes y mezcla bien.

Nota: puedes conservarlo en el frigorífico tapado con film transparente.\\
\hline
%---JUEVES---%
\pbox{20cm}
{
    \rule{0pt}{3ex}\begin{large}\textbf{Jueves}\end{large}\\ 
    \rule{0pt}{2ex}Ensalada caprese\\ con pollo
} & 
\vspace{-0.6cm}
            
\begin{compactitem} 
    \begin{scriptsize}
        \item Bolitas de mozzarella marinadas
        \item Pechuga de pollo
        \item Tomates cherry
        \item Pimienta molida
        \item Sal
        \item Aceite de oliva
    \end{scriptsize}
\end{compactitem} &
\vspace{-0.6cm}     
Saca las bolitas de mozzarella de su marinada y vierte la mitad de la marinada en un bol.
Coloca el pollo en el bol con marinada, sazona con sal y pimienta y déjalo reposar durante 1 hora a temperatura ambiente.
Corta las bolitas de mozzarella, ponlas en un bol aparte y añade la otra mitad de marinada. Reserva.
Añade los tomates cherry al bol de la mozzarella y sazona con sal y pimienta. Reserva.
Cocina la pechuga de pollo a la plancha dejando que se tueste un poco y córtala en pequeños cubos.
Añade los trocitos de pollo al bol con la mozzarella y los tomates, mezcla y sirve.\\
\hline

%---VIERNES---%
\pbox{20cm}
{
    \rule{0pt}{3ex}\begin{large}\textbf{Viernes}\end{large}\\ 
    \rule{0pt}{2ex}Ensalada de brócoli\\ y manzana
} & 
\vspace{-0.6cm}
\begin{compactitem} 
    \begin{footnotesize}
        \item 2 tazas de de brócoli
        \item 1 manzana pelada en cubitos
        \item Nueces tostadas y troceadas
        \item Fruta deshidratada
        \item 3 lonchas de tocino
        \item 1 cucharilla de miel
        \item Vinagre
        \item \nicefrac{1}{2} taza de mayonesa
        \item Sal
        \item Pimienta
        \item Aceite de oliva
    \end{footnotesize}
\end{compactitem}&
\vspace{-0.6cm}
Trocea el tocino y cocínalo a la plancha con unas gotas de aceite de oliva. Reserva.
Introduce el brócoli en agua hirviendo durante un minuto y enfríalo bajo el grifo. Escurre y reserva.
Coloca en un bol el brócoli, la manzana, las nueces, la fruta deshidratada y el tocino. Reserva.
En otro bol, mezcla la miel, el vinagre y la mayonesa hasta conseguir un aderezo de consistencia suave.
Sazona el aderezo con sal y pimienta.
Añade el aderezo al bol con el brócoli y el resto de ingredientes y mezcla bien.

Nota: puedes ajustar las cantidades de ingredientes usados para el aderezo según tus gustos.\\
\hline

%---SABADO---%
\pbox{20cm}
{
    \rule{0pt}{3ex}\begin{large}\textbf{Sabado}\end{large}\\ 
    \rule{0pt}{2ex}Ensalada de tomate,\\ tocino y aguacate
} & 
\vspace{-0.6cm}
\begin{compactitem} 
    \begin{footnotesize}
        \item 1 aguacate maduro
        \item 1 tomate
        \item 2 huevos cocidos
        \item 3 lonchas de tocino
        \item Sal
        \item Pimienta
        \item Zumo de limón
        \item Aceite de oliva
    \end{footnotesize}
\end{compactitem}&
\vspace{-0.6cm}
Cocina el tocino a la plancha con unas gotas de aceite de oliva y córtalo en trozos muy pequeños. Reserva.
Trocea el aguacate, los huevos y el tomate. Reserva.
Exprime uno o dos gajos de limón para obtener el zumo.
Coloca el aguacate, los huevos, el tomate y el tocino en un bol.
Sazona con sal y pimienta y añade el zumo de limón.
Mézclalo todo bien y sirve.\\
\hline

\newpage        
\end{tabular}
\end{document}
