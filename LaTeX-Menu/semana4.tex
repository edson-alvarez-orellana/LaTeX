\documentclass[menu.tex]{subfiles}
\graphicspath{ {images/} }
\begin{document}    
\begin{tabular} {p{3cm} p{4.5cm} p{9cm}} 
\multicolumn{3}{c}{\begin{LARGE}Menú Semanal 4\end{LARGE}}\\
\hline

%---LUNES---%
\pbox{20cm}
{
    \rule{0pt}{3ex}\begin{large}\textbf{Lunes}\end{large}\\ 
    \rule{0pt}{2ex}Revuelto de carne
} & 
\vspace{-0.3cm}            
\begin{compactitem} 
    \begin{scriptsize}
        \item 1 kg carne molida
        \item 4 papas
        \item 2 zanahorias
        \item 3 huevos
        \item 1 taza arvejas
        \item 1 cucharilla de sal
        \item pizca de pimienta
        \item pizca de comino
        \item 2 dientes de ajo
        \item 1 cebolla
        \item 1 taza de arroz
        \item Aceite
    \end{scriptsize}
\end{compactitem}&
\vspace{-0.3cm}
Primeramente freír las papitas así como si estuviera preparando papas fritas, la mejor manera de poderles escurrir el aceite sobrante es poner papel toalla o papel sabana en la fuente que utilizaremos para poner nuestras papitas fritas. Seguidamente picamos nuestra cebollita en cubitos así como los dientes de ajo, estos dos juntos los ponemos a sofreír en una cucharadita de aceite caliente, una ves que estén trasparentes sacarlos en un platillo. Poner a calentar nuevamente la sartén esta vez con dos cucharaditas de aceite, una ves caliente poner la carne y aplastarla con la cuchara de palito, condimentamos nuestra carne al gusto, con la sal, comino y pimienta, si quiere puede añadirle una pizca de orégano también añadir la zanahoria y arvejas , esperar a freír muy bien la carne. Una vez frita, añadir la cebollita y el ajo, mezclar y también añadir las papas fritas, mezclar nuevamente hasta que los ingredientes estén uniformemente mezclados , entonces añadir los tres huevos cubriendo el sartén y todos los ingredientes. Servir con arroz graneado.\\
\hline

%---MARTES---%
\pbox{20cm}
{
    \rule{0pt}{3ex}\begin{large}\textbf{Martes}\end{large}\\ 
    \rule{0pt}{2ex}Ensalada verde
} & 
\vspace{-0.4cm}
\hspace{0.5cm}\begin{footnotesize}Ingredientes (ensalada)\end{footnotesize}
\begin{compactitem} 
    \begin{scriptsize}
        \item Espinacas y otras verduras
        \item 1 aguacate
        \item 1 manzana
        \item Tomates cherry
        \item 2 huevos cocidos
        \item 4 lonchas de tocino
        \item Un puñado de nueces
        \item Aceite de oliva
    \end{scriptsize}
\end{compactitem}
\hspace{0.3cm}
\begin{footnotesize}Ingredientes (aderezo)\end{footnotesize}
\begin{compactitem} 
    \begin{scriptsize}
        \item Zumo de medio limón
        \item 2 cucharadas de mostaza
        \item 2 cucharadas de aceite de oliva
        \item 1 cucharada de vinagre
        \item 1 diente de ajo picado
    \end{scriptsize}
\end{compactitem} &
\vspace{-0.4cm}
Cocina el tocino a la plancha con unas gotas de aceite de oliva. Córtalo en trocitos y reserva.
Corta el aguacate, la manzana, los tomates y los huevos en trocitos. Reserva.
En un bol, coloca primero las espinacas y el resto de verduras verdes al fondo y añade los ingredientes restantes encima. Reserva.
Mezcla todos los ingredientes mencionados para el aderezo y mézclalo bien.
Añade el aderezo al bol de la ensalada y remueve para que empape bien.\\
\hline

%---MIERCOLES---%
\pbox{20cm}
{
    \rule{0pt}{3ex}\begin{large}\textbf{Miércoles}\end{large}\\ 
    \rule{0pt}{2ex}Ensalada de pollo
} & 
\vspace{-0.4cm}            
\begin{compactitem} 
    \begin{scriptsize}
        \item 2 tazas de pollo troceado
        \item 1 taza de uvas rojas (u otro tipo) cortadas a la mitad
        \item 2 huevos cocidos y troceados
        \item Mayonesa (mejor casera)
        \item Un poquito de eneldo fresco
        \item 1 diente de ajo picado
        \item Sal
        \item Pimienta
    \end{scriptsize}
\end{compactitem}&
\vspace{-0.4cm}        
Fríe el pollo en una sartén con aceite de oliva. Reserva.
En un bol, coloca el pollo, añade el resto de ingredientes y mezcla bien.

Nota: puedes conservarlo en el frigorífico tapado con film transparente.\\
\hline

%---JUEVES---%
\pbox{20cm}
{
    \rule{0pt}{3ex}\begin{large}\textbf{Jueves}\end{large}\\ 
    \rule{0pt}{2ex}Ensalada de \\col rizada 
} & 
\vspace{-0.6cm}
\begin{compactitem} 
    \begin{scriptsize}
        \item 2 tazas de col rizada troceada y sin nervios
        \item 1 taza de col roja troceada
        \item 1 taza de zanahoria rallada
        \item 2 manzanas ralladas y sin pelar
        \item Un puñado de pipas peladas
        \item Aceite de oliva
        \item Vinagre de sidra de manzana
        \item Sal
        \item Pimienta molida
        \item Tomillo seco
    \end{scriptsize}
\end{compactitem}&
\vspace{-0.6cm}
En un pequeño bol, mezcla el aceite de oliva, el vinagre, la sal, la pimienta y el tomillo. Reserva.
En un bol más grande, coloca la col rizada, la col, las zanahorias y las manzanas.
Añade el aderezo hecho con el aceite de oliva y el resto de ingredientes al bol de la col rizada y remueve.
Cubre el bol con film transparente y déjalo reposar entre 30 y 45 minutos para que la col rizada se ablande.
Pon la ensalada en el grifo durante un ratito para que se enfríe y justo antes de servir, añade las pipas de girasol por encima.\\
\hline

%---VIERNES---%
\pbox{20cm}
{
    \rule{0pt}{3ex}\begin{large}\textbf{Viernes}\end{large}\\ 
    \rule{0pt}{2ex}Ensalada de \\pollo con \\mango y aguacate
} & 
\vspace{-0.8cm}
\begin{compactitem} 
    \begin{scriptsize}
        \item Lechuga
        \item 2 tazas de pollo desmenuzado
        \item 1 mango pelado
        \item 1 aguacate
        \item \nicefrac{1}{2} cucharilla de chile en polvo
        \item \nicefrac{1}{2} cucharilla de comino
        \item Sal
        \item Pimienta
    \end{scriptsize}
\end{compactitem}&
\vspace{-0.8cm}
Coloca el pollo en bol con una pizca de agua para hidratarlo un poco y caliéntalo en el microondas durante un máximo de 15 segundos.
Mezcla el pollo con el chile en polvo y el comino.
Trocea la lechuga, el mango y el aguacate. Reserva.
En un bol, mezcla el pollo con la lechuga y esparce por encima el mango y el aguacate.
Sazona con sal y pimienta.\\
\hline

%---SABADO---%
\pbox{20cm}
{
    \rule{0pt}{3ex}\begin{large}\textbf{Sábado}\end{large}\\ 
    \rule{0pt}{2ex}Ensalada de \\manzana y apio \\con mayonesa
} & 
\vspace{-0.8cm}
\begin{compactitem} 
    \begin{scriptsize}
        \item \nicefrac{1}{4} taza de mayonesa
        \item 3 cucharadas de vinagre
        \item 2 cucharadas de cebollín
        \item 2 manzanas verdes
        \item 10 tallos de apio
        \item sal y pimienta al gusto
    \end{scriptsize}
\end{compactitem}&
\vspace{-0.8cm}
Pica finamente el apio y corta las manzanas sin pelar en julianas muy finas
Mezcla con el resto de los ingredientes
Sazona con sal y pimienta
Conserva en refrigeración hasta el momento de servir, ¡provecho!\\
\hline
\newpage
\end{tabular}
\end{document}
