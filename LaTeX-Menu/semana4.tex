\documentclass[menu.tex]{subfiles}
\graphicspath{ {images/} }
\begin{document}    
\begin{tabular} {p{3.5cm} p{4cm} p{9cm}}
\multicolumn{3}{c}{\begin{LARGE}Menú Semanal 4\end{LARGE}}\\
\hline

%---LUNES---%
\pbox{20cm}
{
    \rule{0pt}{3ex}\begin{large}\textbf{Lunes}\end{large}\\ 
    \rule{0pt}{2ex}Revuleto de carne
} & 
\vspace{-0.3cm}            
\begin{compactitem} 
    \begin{scriptsize}
        \item 1 kg carne molida
        \item 4 papas
        \item 2 zanahorias
        \item 3 huevos
        \item 1 taza arvejas
        \item 1 cucharilla de sal
        \item pizca de pimienta
        \item pizca de comino
        \item 2 dientes de ajo
        \item 1 cebolla
        \item 1 taza de arroz
        \item Aceite
    \end{scriptsize}
\end{compactitem}&
\vspace{-0.3cm}
Primeramente freír las papitas así como si estuviera preparando papas fritas, la mejor manera de poderles escurrir el aceite sobrante es poner papel toalla o papel sabana en la fuente que utilizaremos para poner nuestras papitas fritas. Seguidamente picamos nuestra cebollita en cubitos así como los dientes de ajo, estos dos juntos los ponemos a sofreír en una cucharadita de aceite caliente, una ves que estén trasparentes sacarlos en un platillo. Poner a calentar nuevamente la sartén esta vez con dos cucharaditas de aceite, una ves caliente poner la carne y aplastarla con la cuchara de palito, condimentamos nuestra carne al gusto, con la sal, comino y pimienta, si quiere puede añadirle una pizca de orégano tb añadir la zanahoria en brunoise y arvejas , esperar a freír muy bien la carne. Una vez frita, añadir la cebollita y el ajo, mezclar y también añadir las papas fritas, mezclar nuevamente hasta que los ingredientes estén uniformemente mezclados , entonces añadir los tres huevos cubriendo el sartén y todos los ingredientes. Servir con arroz graneado.\\
\hline

%---MARTES---%
\pbox{20cm}
{
    \rule{0pt}{3ex}\begin{large}\textbf{Martes}\end{large}\\ 
    \rule{0pt}{2ex}plato
} & 
\vspace{-0.1cm}            
\begin{compactitem} 
    \begin{scriptsize}
        \item x
    \end{scriptsize}
\end{compactitem}&
\vspace{-0.1cm}            
Receta.\\
\hline

%---MIERCOLES---%
\pbox{20cm}
{
    \rule{0pt}{3ex}\begin{large}\textbf{Martes}\end{large}\\ 
    \rule{0pt}{2ex}plato
} & 
\vspace{-0.1cm}                      
\begin{compactitem} 
    \begin{scriptsize}
        \item x
    \end{scriptsize}
\end{compactitem}&
\vspace{-0.1cm}            
Receta.\\

\hline

%---JUEVES---%
\pbox{20cm}
{
    \rule{0pt}{3ex}\begin{large}\textbf{Martes}\end{large}\\ 
    \rule{0pt}{2ex}plato 
} & 
\vspace{-0.1cm}                       
\begin{compactitem} 
    \begin{scriptsize}
        \item x
    \end{scriptsize}
\end{compactitem}&
\vspace{-0.1cm}            
Receta.\\
\hline

%---VIERNES---%
\pbox{20cm}
{
    \rule{0pt}{3ex}\begin{large}\textbf{Martes}\end{large}\\ 
    \rule{0pt}{2ex}plato
} & 
\vspace{-0.1cm}            
\begin{compactitem} 
    \begin{scriptsize}
        \item x
    \end{scriptsize}
\end{compactitem}&
\vspace{-0.1cm}
Receta.\\
\hline

%---SABADO---%
\pbox{20cm}
{
    \rule{0pt}{3ex}\begin{large}\textbf{Martes}\end{large}\\ 
    \rule{0pt}{2ex}plato
} & 
\vspace{-0.1cm}            
\begin{compactitem} 
    \begin{scriptsize}
        \item x
    \end{scriptsize}
\end{compactitem}&
\vspace{-0.1cm}
Receta.\\
\hline
\newpage
\end{tabular}
\end{document}
