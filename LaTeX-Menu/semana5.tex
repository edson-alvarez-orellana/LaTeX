\documentclass[menu.tex]{subfiles}
\graphicspath{ {images/} }
\begin{document}    
\begin{tabular} {p{3cm} p{4.5cm} p{9cm}}
\multicolumn{3}{c}{\begin{LARGE}Menú Semanal 5\end{LARGE}}\\
\hline

%---LUNES---%
\pbox{20cm}
{
    \rule{0pt}{3ex}\begin{large}\textbf{Lunes}\end{large}\\ 
    \rule{0pt}{2ex}Ensalada mixta
}& 
\vspace{-0.3cm}
\begin{compactitem} 
    \begin{scriptsize}
        \item Pechuga de pollo troceada
        \item 1 palta maduro pelado
        \item 1 manzana pelada
        \item \nicefrac{1}{4} taza de apio
        \item \nicefrac{1}{2} taza de cebolla
        \item Perejil o cilantro
        \item 2 cucharillas de limón
        \item Sal
        \item Pimienta negra molida
        \item Aceite de oliva
    \end{scriptsize}
\end{compactitem}&
\vspace{-0.3cm} 
Mezcla repollo picado fino y manzana picada fina con cebolla y tomate en cuadros. Agregar sal y aceite a gusto, usa esta preparación para acompañar la carne de cualquier tipo cuidando de retirar la grasa de la misma pues la fruta mezclada con grasa es una inadecuada combinación.\\
\hline

%---MARTES---%
\pbox{20cm}
{
    \rule{0pt}{3ex}\begin{large}\textbf{Martes}\end{large}\\ 
    \rule{0pt}{2ex}Pollo crocante
}& 
\vspace{-0.4cm}            
\begin{compactitem} 
    \begin{scriptsize}
        \item 1 taza de fresas troceadas
        \item 1 taza de manzana troceada
        \item 1 taza de uvas en mitades
        \item 1 taza de naranja troceada
        \item 2 aguacates troceados
        \item 1 taza de nueces
        \item 1 taza de pechuga de pollo
        \item 1 taza de tocino troceado
        \item 1 taza de cebolla troceada
        \item Queso rallado
        \item Aceite de oliva               
    \end{scriptsize}
\end{compactitem}&
\vspace{-0.4cm}
Mezcle las galletas picadas (muy finamente) con la pimienta, el comino, la sal y las claras de huevo cubrir las presas de pollo y cocinarlas.\\
\hline

%---MIERCOLES---%
\pbox{20cm}
{
    \rule{0pt}{3ex}\begin{large}\textbf{Miércoles}\end{large}\\
    \rule{0pt}{2ex}Salteado de palmito
}&
\vspace{-0.4cm}
\begin{compactitem} 
    \begin{footnotesize}
        \item Tortillas integrales o Lechuga
        \item Carne molida
        \item Tomate
        \item Cebolla
        \item Palta
    \end{footnotesize}
\end{compactitem}&
\vspace{-0.4cm}
En el sartén con aceite en aerosol o aceite de oliva, colocar ajo, tomate y cebolla picados, seguidamente saltear estos ingredientes. Agregar repollo, rábano, acelga, palmito y jamón de pollo en cuadrados. Continuar salteando y finalmente agregar queso y orégano a la preparación.\\
\hline

%---JUEVES---%
\pbox{20cm}
{
    \rule{0pt}{3ex}\begin{large}\textbf{Jueves}\end{large}\\
    \rule{0pt}{2ex}Sopa de huevo
} & 
\vspace{-0.4cm}
\begin{compactitem} 
    \begin{footnotesize}
        \item Carote berenjenas o brócoli
        \item Huevo
        \item Jamón
        \item Queso
    \end{footnotesize}
\end{compactitem}&
\vspace{-0.4cm}
Freír verduras variadas (zanahoria, vainas, espinacas con una cantidad mayor de achojcha). Hervir con carne magra, colocar claras de huevos o huevo completo si está permitido en su sistema.\\
\hline

%---VIERNES---%
\pbox{20cm}
{
    \rule{0pt}{3ex}\begin{large}\textbf{Viernes}\end{large}\\ 
    \rule{0pt}{2ex}Souffle de espinacas
} & 
\vspace{-0.3cm}
\begin{compactitem} 
    \begin{footnotesize}
        \item \nicefrac{1}{2} kg de trigo pelado
        \item \nicefrac{1}{2} kg de papas
        \item \nicefrac{1}{4} kg carne de res
        \item Cuerito de cerdo un pedazo
        \item 2 cucharas de colorante rojo
        \item \nicefrac{1}{2} taza de habas
        \item \nicefrac{1}{4} taza de arvejas
        \item 2 cebollas medianas
        \item 1 zanahoria rallada
        \item 2 ramitas de perejil picado
        \item 2 dientes de ajo molidos
        \item \nicefrac{1}{4} cucharilla de comino
        \item 4 cucharas de aceite
        \item 6 litros de agua
        \item Sal a gusto
        \item 1 cuchara de perejil picado
    \end{footnotesize}
\end{compactitem}&
\vspace{-0.3cm}
Mezcla una mano de espinacas picadas a corte fino con 4 claras de huevos, sal, 4 tiras de queso y Jamón (todo picado). Cocinar en la sartén con teflón recubierto con una capa delgada de aceite en aerosol tipo PAM o aceite de oliva.\\
\hline

%---SABADO---%
\pbox{20cm}
{
    \rule{0pt}{3ex}\begin{large}\textbf{Sábado}\end{large}\\
    \rule{0pt}{2ex}Tortilla de atun
}& 
\vspace{-0.3cm}
\begin{compactitem} 
    \begin{footnotesize}
        \item Verduras
        \item Huevo
        \item Queso
        \item Jamón o enrollado de pollo
        \item Aceitunas
        \item Carne en trocitos de pollo
        \item Salsa cesar light
    \end{footnotesize}
\end{compactitem}&
\vspace{-0.3cm}
Mezclar 1 lata de atún al agua con 3 a 4 claras de huevo. Agregar hongos picados. Llevar al microondas hasta que la preparación se haga sólida alcanzando la consistencia de una tortilla.\\ \hline
\newpage
\end{tabular}
\end{document}
