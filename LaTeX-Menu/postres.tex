\documentclass[menu.tex]{subfiles}
\graphicspath{ {images/} }
\begin{document}    
\begin{tabular} {p{3cm} p{4.5cm} p{9cm}}
\multicolumn{3}{c}{\begin{LARGE}Postres\end{LARGE}}\\
\hline

\pbox{20cm}
{
    \rule{0pt}{2ex}Crema cafe
}& 
\vspace{-0.1cm}
\begin{compactitem} 
    \begin{scriptsize}
        \item Yogurt natural
        \item Stevia
        \item Cda. Cafe
    \end{scriptsize}
\end{compactitem}&
\vspace{-0.1cm}
Mezcla de yogurt natural con una pizca de Stevia y 1 cucharada de café, Si desea un postre con mayor consistencia licua la preparación.\\
\hline

\pbox{20cm}
{
    \rule{0pt}{2ex}Gelatina de \\diversos sabores
}& 
\vspace{-0.1cm}            
\begin{compactitem} 
    \begin{scriptsize}
        \item Gelatina sin sabor
        \item 1 sobre refresco de dieta tipo livean
    \end{scriptsize}
\end{compactitem}&
\vspace{-0.1cm}
Mezclar polvo de gelatina neutra (para litro) con 1 sobre de refresco en polvo de dieta tipo livean. Hacer la dilución habitual con agua caliente y fría, cuajar normalmente.\\
\hline

\pbox{20cm}
{
    \rule{0pt}{2ex}Jugo de \\frutas con crema
}&
\vspace{-0.1cm}
\begin{compactitem} 
    \begin{footnotesize}
        \item Fruta
        \item Crema para café sin sabor
    \end{footnotesize}
\end{compactitem}&
\vspace{-0.1cm}
Licua tu fruta favorita con crema para café (neutra o sin sabor), tipo Non Dairy Creamer.\\
\hline

\pbox{20cm}
{
    \rule{0pt}{2ex}Mousse de gelatina\\ con crema
} & 
\vspace{-0.1cm}
\begin{compactitem} 
    \begin{footnotesize}
        \item Gelatina light
        \item Crema para café
    \end{footnotesize}
\end{compactitem}&
\vspace{-0.1cm}
Prepara la gelatina Light normalmente, coloca encima 2 cucharas de crema para café neutra o sin sabor (Non Dairy Creamer) diluidas previamente con 1 cuchara de agua.\\
\hline

\pbox{20cm}
{
    \rule{0pt}{2ex}Pomelo batido
} & 
\vspace{-0.1cm}
\begin{compactitem} 
    \begin{footnotesize}
        \item Jugo de pomelo
        \item Huevo
        \item Edulcorante o Stevia
    \end{footnotesize}
\end{compactitem}&
\vspace{-0.1cm}
Licuar 2 vasos de jugo de pomelo con 1 clara de huevo y una pizca de edulcorante artificial o Stevia.\\
\hline

\pbox{20cm}
{
    \rule{0pt}{2ex}Crema de huevos
}& 
\vspace{-0.1cm}
\begin{compactitem} 
    \begin{footnotesize}
        \item Huevo
        \item Polvo de refresco Livean
    \end{footnotesize}
\end{compactitem}&
\vspace{-0.3cm}
Batir claras de huevo (frías) a punto nieve. Agregar el polvo de refresco LIVEAN hasta obtener el sabor deseado (No precisa Stevia).\\ 
\hline

\pbox{20cm}
{
    \rule{0pt}{2ex}Huevos batidos
}& 
\vspace{-0.1cm}
\begin{compactitem} 
    \begin{footnotesize}
        \item Huevo
        \item Coca-Cola Zero
    \end{footnotesize}
\end{compactitem}&
\vspace{-0.1cm}
Batir claras de huevo (Frías) a punto nieve, agregar Stevia y luego Coca Cola Zero.\\ 
\hline

\pbox{20cm}
{
    \rule{0pt}{2ex}Budín de yogurt
}& 
\vspace{-0.1cm}
\begin{compactitem} 
    \begin{footnotesize}
        \item Gelatina dietética
        \item Gelatina sin sabor
        \item Yogurt
        \item Polvo de refresco livean
        \item Coco rallado
    \end{footnotesize}
\end{compactitem}&
\vspace{-0.1cm}
Mezclar polvo de gelatina dietética (para 1l.) con cucharilla de gelatina sin sabor. Agregar 2 vasos de agua caliente, 1 de agua fría y uno de yogurt previamente endulzado con el polvo LIVEAN, refrigerar, se puede colocar una delicada capa de coco rallado en la superficie. Una vez cuajado puede congelarse para obtener un helado cremoso.\\ 
\hline

\pbox{20cm}
{
    \rule{0pt}{2ex}Flan de yogurt
}& 
\vspace{-0.1cm}
\begin{compactitem} 
    \begin{footnotesize}
        \item Budín de yogurt
        \item Yogurt de frutilla
        \item Gelatina de cerezas
        \item Gelatina de cerezas
    \end{footnotesize}
\end{compactitem}&
\vspace{-0.1cm}
Hacer el budín de yogurt como indica la receta correspondiente. La variación radica en usar 2 vasos de agua caliente y 2 vasos de yogurt de frutilla. Entonces agreguemos a la preparación la gelatina de cerezas (ya cuajada), picada en cuadritos, volver a cuajar la receta ya completa.\\ 
\hline

\pbox{20cm}
{
    \rule{0pt}{2ex}Helado cremoso\\ de frutilla
}& 
\vspace{-0.1cm}
\begin{compactitem} 
    \begin{footnotesize}
        \item \nicefrac{1}{2} litro de yogurt natural
        \item Polvo Livean
        \item \nicefrac{1}{4} caja de frutillas
        \item \nicefrac{1}{2} taza de leche deslactosada
    \end{footnotesize}
\end{compactitem}&
\vspace{-0.1cm}
Licuar \nicefrac{1}{2} litro de yogurt natural con polvo LIVEAN (la cantidad necesaria para que el preparado tenga sabor intenso), agregar \nicefrac{1}{4} caja de frutillas, \nicefrac{1}{2} taza de leche deslactosada. Congelar batiendo de tiempo en tiempo para que la preparación no enhiele y quede cremosa.\\ 
\hline

\pbox{20cm}
{
    \rule{0pt}{2ex}Helado de piña
}& 
\vspace{-0.1cm}
\begin{compactitem} 
    \begin{footnotesize}
        \item Piña
        \item Hierba buena
        \item Polvo livean
    \end{footnotesize}
\end{compactitem}&
\vspace{-0.1cm}
Licuar 1 trozo de piña con agua y 4 hojas de hierbabuena. Mezclar con polvo LIVEAN (la cantidad necesaria para que el preparado tenga un sabor intenso). Batir a mano la preparación y congelar. Si desea puede tomarse como refresco agregando hielo.\\ 
\hline

\newpage
\end{tabular}
\end{document}
