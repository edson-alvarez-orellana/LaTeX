%% Ensalada César

\begin{otherlanguage}{spanish}

\setHeadlines
{% translation
    inghead = Ingredientes,
    prephead = Preparacion,
    hinthead = Nota,
    continuationhead = Continuacion,
    continuationfoot = Pie de continuacion,
    portionvalue = Personas,
}

\begin{recipe}
[ % Optionale Eingaben
    preparationtime = {\unit[5]{min}},
    portion = \portion{2},
    %source = Restaurante Miraflores
]
{Ensalada Cesar}
    
    \graph
    {% Imagen
        %small=pic/glass,    % kleines Bild
        big=pic/ensalada-cesar % groes (lngeres) Bild
    }
    
    \ingredients
    {% Ingredientes
        \unit [200]{g} & lechuga romana\\
        & Salsa César\\
        & Picastostes de pan\\
        \unit [40]{g} & queso parmesano\\
        \unit [\nicefrac{1}{2}]{u} & pechuga de pollo\\
        & Sal\\
        & Jamon\\
        & Pimienta negra
    }
    \preparation
    { % Preparacion
        \step Salpimentamos la pechuga de pollo y hacemos a la plancha, reservamos.
        \step Lavamos la lechuga, podéis utilizar la variedad que os guste, pero si queréis hacer la más típica utilizad lechuga romana. La secamos y picamos finamente.
        \step Ponemos en la fuente o plato la lechuga, encima la pechuga cortada en trozos pequeños o en tiras, unos picatostes de pan y unos trocitos de queso parmesano.
        \step Incorporamos un poco de salsa, mezclamos y rallamos por encima queso parmesano.
    }

\end{recipe}

\end{otherlanguage}
