%% Tacos en base de tortillas y en base vegetal

\begin{otherlanguage}{spanish}

\setHeadlines
{% translation
    inghead = Ingredientes,
    prephead = Preparación,
    hinthead = Nota,
    continuationhead = Continuación,
    continuationfoot = Pie de continuación,
    portionvalue = Personas,
}

\begin{recipe}
[ % Optionale Eingaben
    preparationtime = {\unit[1]{h}},
    portion = \portion{5-6},
    source = doña Celia la Fuente Peredo
]
{Silpancho}
    
    \graph
    {% Imagen
        %small=pic/glass,    % kleines Bild
        big=pic/silpancho % groes (lngeres) Bild
    }
    
    \ingredients
    {% Ingredientes
        & Carne en filetes\\
        \unit [1]{taza} & de arroz graneado\\
        4 & papas cocidas\\
        4 & huevos\\
        & Tomate, Cebolla y Locoto\\
        & Pan molido\\
        & Vinagre, Aceite\\
        & Sal y Pimienta
    }
    
    \preparation
    { % Preparacion
        \step Sazonar la carne con sal y pimienta y adelgazar con el pan molido hasta que quede como un asado grande y delgado, se debe agregar pan molido constantemente y suficiente para que la carne no se rompa con los golpes. 
        \step Freir las papas previamente cocidas y cortadas en rodajas, los huevos y la carne apanada. 
        \step Picar el tomate, la cebolla y el locoto en cuadritos chicos; mezclar y sazonar con sal, pimienta y vinagre.
        \step Servir en un plato poniendo una cama de arroz, distribuir unas cuantas papas, encima poner la carne, el huevo frito y decorar con la salsa cruda.
    }
    
    \hint
    {% Notas
        Servir como en la foto.
    }

\end{recipe}

\end{otherlanguage}

