%% Pique Macho

\begin{otherlanguage}{spanish}

\setHeadlines
{% translation
    inghead = Ingredientes,
    prephead = Preparación,
    hinthead = Nota,
    continuationhead = Continuación,
    continuationfoot = Pie de continuación,
    portionvalue = Personas,
}

\begin{recipe}
[ % Optionale Eingaben
    preparationtime = {\unit[1.5]{h}},
    portion = \portion{5-6},
    source = Restaurante Miraflores
]
{Pique Macho}
    
    \graph
    {% Imagen
        %small=pic/glass,    % kleines Bild
        big=pic/pique_macho % groes (lngeres) Bild
    }
    
    \ingredients
    {% Ingredientes
        \unit [1]{Kg} & Carne de res\\
        \unit [\nicefrac{1}{2}]{Kg} & Salchichas\\
        8 & Papas medianas\\
        4 & Cebollas picadas tipo pluma\\
        2 & Tomates medianos\\
        2 & Locotos picados\\
        \unit [1]{cda.} & Pimienta molida\\
        \unit [1]{pisca} & Comino molido\\
        \unit [2]{cda.} & Sal\\
        & Ajo molido (opcional)\\
        3 & Huevos cocidos\\
        \unit [1]{\nicefrac{1}{2}} & Taza de aceite
    }
    
    \preparation
    { % Preparacion
        \step Cortar la carne en dados pequeños y condimentarla con sal, ajo, pimienta y comino a gusto.
        \step Freír la carne tapando la cacerola para obtener una carne jugosa.
        \step Aparte, freír las salchichas cortadas en rodajas.
        \step Pelar y cortar las papas en bastones y freírlas en aceite caliente.
        \step Mezclar la carne, las papas fritas y las salchichas.
        \step Servir adornando con el locoto crudo picado en tiras largas, los huevos duros, la cebolla y el tomate.
    }
    
    \hint
    {% Notas
        Añadir ketchup, Mostaza y Mayonesa a gusto.
    }

\end{recipe}

\end{otherlanguage}
