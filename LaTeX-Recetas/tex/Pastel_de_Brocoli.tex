%% Ensalada César

\begin{otherlanguage}{spanish}

\setHeadlines
{% translation
    inghead = Ingredientes,
    prephead = Preparacion,
    hinthead = Nota,
    continuationhead = Continuacion,
    continuationfoot = Pie de continuacion,
    portionvalue = Personas,
}

\begin{recipe}
[ % Optionale Eingaben
    preparationtime = {\unit[10]{min}},
    portion = \portion{4},
    calory = {\unit[350] {Cal}},
    bakingtime = {\unit[17]{min}},
    source = Cocina Española
]
{Pastel de brócoli al microondas}
    
    \graph
    {% Imagen
        %small=pic/glass,    % kleines Bild
        big=pic/pastel-de-brocoli % groes (lngeres) Bild
    }
    
    \ingredients
    {% Ingredientes
        \unit [500]{g} & de brocoli cocido\\
        \unit [150]{g} & de queso cheddar\\
        3 & huevos\\
        \unit [200]{ml} & de leche\\
        & Aceite de Oliva\\
        & Sal a gusto\\
        & Pizca de pimienta negra
    }
    \preparation
    { % Preparacion
        \step Lo primero que haremos, será cocer el brócoli si no lo tenemos cocido. Bastará con añadirlo a una olla, con agua que lo cubra completamente y sal. Dejamos hervir durante aproximadamente 5 minutos.
        \step Mientras, vamos a untar de aceite de oliva el recipiente que vayamos a utilizar nuestro pastel de brócoli. De esta manera, nos aseguramos que no se nos adhiera a las paredes y podamos sacarlo para emplatar fácilmente.
        \step Lo siguiente, será coger un bol y añadir los 3 huevos, los 200 ml de leche (nosotros hemos usado semidesnatada normal), una pizca de sal como de pimienta negra molida. Batimos bien hasta conseguir una mezcla homogénea.
        \step Cuando el brócoli haya hervido, lo vamos a ir introduciendo en el recipiente de manera equitativa y a continuación, echamos la mezcla anterior que hemos batido en el bol. Lo último que añadiremos serán los taquitos de queso cheddar. Los repartiremos bien por todo el recipiente (no sólo por la superficie sino también por dentro).
        \step Y lo último será introducirlo en el microondas, a máxima potencia, alrededor de los 15-17 minutos.
    }

\end{recipe}

\end{otherlanguage}
