%% Llusp’ichi

\begin{otherlanguage}{spanish}

\setHeadlines
{% translation
    inghead = Ingredientes,
    prephead = Preparación,
    hinthead = Nota,
    continuationhead = Continuación,
    continuationfoot = Pie de continuación,
    portionvalue = Personas,
}
    \begin{recipe}
        [ % Optionale Eingaben
            preparationtime = {\unit[1]{h}},
            portion = \portion{5},
            %source = Restaurant Miraflores
        ]
        {Llusp’ichi}
        
        \graph
        {% Imagen
            %small=pic/glass,    % kleines Bild
            big=pic/lluspichi % groes (lngeres) Bild
        }
        
        \ingredients
        {% Ingredientes
            \unit [\nicefrac{1}{2}]{Kg} & de trigo pelado\\
            \unit [\nicefrac{1}{2}]{Kg} & de papas\\
            \unit [\nicefrac{1}{4}]{Kg} & carne de res\\
            & Cuerito de cerdo un pedazo\\
            \unit [2]{cda.} & de colorante rojo\\
            \unit [\nicefrac{1}{2}]{taza} & de habas\\
            \unit [\nicefrac{1}{4}]{taza} & de arvejas\\
            2 & cebollas medianas\\
            1 & zanahoria rallada\\
            2 & ramitas de perejil picado\\
            2 & dientes de ajo molidos\\
            \unit [\nicefrac{1}{4}]{cda} & de comino\\
            4 & cucharas de aceite\\
            \unit [6]{lt} & de agua\\
            1 cda. & Perejil picado
        }
        
        \preparation
        { % Preparacion
            \step Remojar el trigo pelado noche antes. Al dia siguiente lavar en varias aguas.
            \step Poner al fuego una olla con el agua. Cuando empiece a hervir a˜nadir las carne de res, el cuerillo de cerdo y el trigo. Aparte calentar el aceite en la sarten en el fuego y sofreir la cebolla con el ajo.
            \step Dejando dorar, incorporar el comino, el perejil, la zanahoria, el colorante y las arvejas. A˜nadir este ahogado a la olla y dejar cocer.
            \step Mover continuamente para que no se pegue al fondo de la olla, hasta que las carnes esten suaves y el trigo reviente, luego incorporar las papas, las habas. 
            \step Sazonar con la sal y el oregano, al final si esta muy liquido a˜nadir un poco de harina diluida en agua sin grumos. Dejar cocer las papas y retirar del fuego.
        }
        
        \hint
        {% Notas
            Servir caliente.
        }

    \end{recipe}

\end{otherlanguage}
