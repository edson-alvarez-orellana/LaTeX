%% Ensalada de pollo con Palta

\begin{otherlanguage}{spanish}

\setHeadlines
{% translation
    inghead = Ingredientes,
    prephead = Preparación,
    hinthead = Nota,
    continuationhead = Continuación,
    continuationfoot = Pie de continuación,
    portionvalue = Personas,
}
    \begin{recipe}
        [ % Optionale Eingaben
            preparationtime = {\unit[1]{h}},
            portion = \portion{5},
            %source = Restaurant Miraflores
        ]
        {Ensalada de pollo con Palta}
        
        \graph
        {% Imagen
            %small=pic/glass,    % kleines Bild
            big=pic/ensalada-pollo-mango-palta % groes (lngeres) Bild
        }
        
        \ingredients
        {% Ingredientes
            1 & Pechuga de pollo troceada\\
            1 & palta maduro pelado\\
            1 & manzana pelada\\
            \unit [\nicefrac{1}{4}]{taza} & de apio\\
            \unit [\nicefrac{1}{2}]{taza} & de cebolla\\
            & Perejil o cilantro\\
            \unit [2]{cda.} & de limon\\
            & Sal\\
            & Pimienta negra molida\\
            & Aceite de oliva
        }
        
        \preparation
        { % Preparacion
            \step Corta la pechuga de pollo en trozos peque˜nos y sofrie en una sarten con un poco de aceite de oliva. Reserva.
            \step Trocea el aguacate, la manzana, el apio y la cebolla. Reserva.
            \step Coloca en un bol los trocitos de pollo, aguacate, manzana, apio y cebolla.
            \step Con la ayuda de un tenedor machaca el aguacate y mezclalo con el resto de ingredientes hasta formar una especie de pasta.
            \step A˜nade el perejil (o cilantro), el zumo de limon (o lima), la sal y la pimienta y mezclalo bien.
            \step Puedes a˜nadir mas zumo o incluso aceite de oliva si ves que esta demasiado seco.
            \step Sirve acompa˜nado de pan, pan tostado o algun tipo de hoja verde para ensalada, como lechuga, berros, verdolaga.
        }
        
        \hint
        {% Notas
            Puedes conservarlo en el frigorifico tapado con film transparente..
        }

    \end{recipe}

\end{otherlanguage}
